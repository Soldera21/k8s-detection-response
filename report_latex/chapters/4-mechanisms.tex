\section{Rules and Logs of Falco and Tracee}
\label{sec:rules}

In this section, we discuss in more details how the rules work in Falco and Tracee and how their logs can be interpreted. What we want to highlight here are the differences in the rules engines and the information that is included in the event produced after detection.


\subsection{Falco}
The rule engine of Falco is based on the YAML format like other standard configuration files in a Kubernetes environment. This enables the knowledge of a single formatting language that is then adapted to the specific knowledge of how Falco rules are made. We report here the most interesting fields of a rule. More details can be found in \cite{falcoFields}:
\begin{itemize}
    \item \textbf{rule}: name for the rule
    \item \textbf{condition}: filtering expression applied to the rule
    \item \textbf{desc}: long description of what is detected
    \item \textbf{output}: message if a matching event occurs
    \item \textbf{priority}: representation of severity between emergency, alert, critical, error, warning, notice, informational, debug
    \item \textbf{exceptions}: special cases where the rule is not applied
\end{itemize}
All the conditions in Falco are composed by a set of default macro conditions\cite{falcoMacros}. Other macros can be overridden to fit the particular configuration of the environment (e.g. SSH port, ...). To compose a rule there is the need to study which macros and logic expressions can match the potential threat we want to monitor. Macros and list can help generalize rules and adding specific configurations for the single system without changing directly the set of rules.

The default set of rules\cite{falcoRules} is based on syscall events and it maintained by the community and Falco authors. They are in different development state from incubating to deprecated and they can create different alert levels from \texttt{INFO} to \texttt{CRITICAL} based on the severity of the event. By default only the stable rules are used. Here we report some examples:
\begin{itemize}
    \item Drop and execute a new binary in a container: a new executable file is injected into the container and it was not present during image build;
    \item PTRACE anti-debug attempt: the PTRACE syscall is often used by malware to understand if they are debugged to mask their behavior;
    \item Redirect STDOUT/STDIN to network connection in container: detects wheter input or output is redirected to the network to alert about potential reverse shell or remote code execution;
    \item Create symlink over sensitive files: detect if a symlink is created for sensitive files in \texttt{/etc} or root directory;
    \item Linux kernel module injection detected: find if a new kernel module is injected using \texttt{insmod} or \texttt{modprobe}.
\end{itemize}


\subsubsection{Log Structure}
To be able to understand what is happening in our system we need to interpret the generated logs. We choose to setup Falco to output events in JSON format. This is used also to forward them to the webhook. Here we reported the most interesting fields of the log:
\begin{itemize}
  \item \textbf{output}: a human-readable message generated by Falco describing the detected behavior;
  \item \textbf{container.id}: the ID of the container where the event was detected;
  \item \textbf{container.name}: the name of the container involved in the event;
  \item \textbf{evt.type}: the type of system call or event detected (e.g. \texttt{open}, \texttt{execve}, \texttt{mkdir});
  \item \textbf{network info}: all the information about IP, port and network protocol used;
  \item \textbf{k8s.ns.name}: the Kubernetes namespace where the pod is running;
  \item \textbf{k8s.pod.name}: the name of the pod where the alert was triggered;
  \item \textbf{proc.name}: the name of the process that triggered the alert (e.g. \texttt{bash}, \texttt{python3});
  \item \textbf{user.name}: the username under which the process was executed (e.g. \texttt{root});
  \item \textbf{priority}: the severity level of the alert (\texttt{Critical}, \texttt{Error}, \texttt{Warning}, \texttt{Notice}, \texttt{Info});
  \item \textbf{rule}: the name of the Falco rule that was triggered;
  \item \textbf{time}: timestamp of when the event occurred (in UTC, RFC3339 format).
\end{itemize}
Logs from Falco and Falco handler can be taken with:

\texttt{bash falco-conf/falco-logs.sh}

\texttt{kubectl logs deployment/falco-handler -n default}

\noindent Here is an example log line during a reverse shell attack. The fields below are only the most interesting in our opinion:
\begin{verbatim}
{
   "output": "08:49:05.463393946: Notice Redirect stdout/stdin to network connection ...",
   "output_fields": {
      "container.id": "0123456789abc",
      "container.name": "k8s_zipapp_zipapp-...",
      "evt.type":"dup2",
      "fd.l4proto": "tcp",
      "fd.sip": "18.158.58.205",
      "k8s.ns.name": default,
      "k8s.pod.name": zipapp-...,
      "proc.aname[2]": "sh",
      "proc.aname[3]": "zip",
      "proc.cmdline": "bash evil ...",
      "proc.name": "bash",
      "user.name": "root",
   },
   "priority":"Notice",
   "rule":"Redirect STDOUT/STDIN to Network Connection in Container",
   "time":"2025-05-28T08:49:05.463393946Z"
}
\end{verbatim}


\subsection{Tracee}
By default, Tracee monitors a set of Linux kernel events that are commonly linked to suspicious or malicious activity. These events are based on system calls and other kernel-level actions that can indicate an attack or unauthorized behavior in containerized environments. It supports the Rego-based rules that are written using the Open Policy Agent (OPA) language. Rego is a declarative, logic-based language commonly used for policy enforcement. In Tracee it enables behavioral detection and granular definition of security policies. Rego rules are checked on every event and they can be hot-reloaded without restarting the engine. Tracee supports both built-in rules and user-defined ones.

Rules for Tracee can be also signature-based written in JSON or Go plugins compiled and loaded in the Tracee engine. These three methods are not completely interchangeable, but they are complementary. Signature-base are simpler and they support not very complex logic. Rego-based rule support a complex logic, but they are difficult to understand and create. The most efficient way to define a rule is with Go plugin that are not hot-reloadable, but they are compiled and loaded. In general we can say that custom rules in Tracee are more complex than Falco ones. This leads to a steeper learning curve and at the same time makes Tracee a forensics tool while Falco is a simpler alerting tool.

Tracee offers a wide set of default rules\cite{traceeRules} that are grouped into the following categories:
\begin{itemize}
    \item \textbf{Security Events}: monitors system activities in real-time, generating security events that equip users with the information they need to assess and respond to the state of their digital environments. For example \texttt{Anti-Debugging}, \texttt{Illegitimate shell} and \texttt{Disk Mount}.
    \item \textbf{Network Events}: makes it easy to trace network activity in common protocols such as \texttt{net\_packet\_ipv4}, \texttt{net\_packet\_tcp}.
    \item \textbf{Syscall}: traces Linux system calls such as \texttt{open}, \texttt{exit}, \texttt{kill}.
    \item \textbf{Extra Events}: other events, that are not included in the other categories, can be monitored such as \texttt{cgroup\_mkdir}, \texttt{file\_modification}.
\end{itemize}


\subsubsection{Log Structure}
Once Tracee is running in the Kubernetes cluster, it begins capturing and logging events related to system activity. The output logs are structured as JSON lines, where each line represents an event captured by eBPF. Each event typically contains the following fields:
\begin{itemize}
    \item \textbf{timestamp}: exact time the event occurred;
    \item \textbf{processId}: ID of the process;
    \item \textbf{userId}: ID of the user executing the process;
    \item \textbf{processName}: name of the process involved in the event;
    \item \textbf{hostName}: name of the pod where the event occurred;
    \item \textbf{eventName}: name of the detected event;
    \item \textbf{syscall}: the type of system call or event detected (e.g. \texttt{open}, \texttt{execve}, \texttt{mkdir});
    \item \textbf{args}: a list of arguments or parameters relevant to the event like network options;
    \item \textbf{metadata}: a detailed description of what happened and a MITRE classification of the event.
\end{itemize}
Logs from Tracee and Tracee handler can be taken with:

\texttt{bash tracee-conf/tracee-logs.sh}

\texttt{kubectl logs deployment/tracee-handler -n default}

\noindent Here is an example log line during a reverse shell attack. The following are the most relevant fields of the log:
\begin{verbatim}
{
   "timestamp": 1748422408882744815,
   "processId": 452,
   "userId": 0,
   "processName": "bash",
   "hostName": "zipapp-5485c4dc",
   "eventName": "stdio_over_socket",
   "syscall": "dup",
   "args":[
      {
         "name": "Port",
         "value": "19023"
      },
      {
         "name": "IP address",
         "value": "3.67.161.133"
      },
      ...
   ],
   "metadata": {
      "Description": "A process has its standard input/output redirected to a socket...",
      "Properties": {
         "Category": "execution",
         "Severity": 3,
         "Technique": "Unix Shell",
         "external_id": "T1059.004",
         "signatureName": "Process standard input/output over socket detected"
      }
   }
}
\end{verbatim}

